\documentclass[8pt,twocolumn]{extarticle}
\usepackage{rotating}
\usepackage{nicefrac}
\usepackage{preamble}


\title{30 Marco}
\author{יואב מרקו}

\begin{document}
‏\section{משפט הגבול המרכזי}
יהיו $X_{1},...,X_{n}$ מ“מ ב“ת שווי התפלגות בעלי שונות
סופית, אזי המ“מ המוגדר $\widehat{S_{n}}=\frac{\sum_{i=1}^{n}X_{i}-n\EX\left[X_{1}\right]}{\sqrt{n\Var\left(X_{1}\right)}}$
(תוחלתו 0, שונות 1) מתכנס בהתפלגות להתפלגות גאוסיאנית, כלומר:

\[\forall -\infty\leq a<b\leq\infty.\
  P\left(a\leq\widehat{S_{n}}\leq b\right)=\frac{1}{\sqrt{2\pi}}\int_{a}^{b}e^{\frac{-x^{2}}{2}}dx=\varPhi\left(b\right)-\varPhi\left(a\right)\]

‏\section{אי שוויונות} ‏\begin{claim}[א"ש מרקוב]
  אם ‎\(X\geq 0\) מ"מ עם תוחלת סופית אז לכל ‎\(t>0\) מתקיים
  % manual linebreak, otherwise breqn breaks it at the t) which is damn ugly
  ‎\(P(X\geq t)\\ \leq  \frac{\EX[X]}{t}\).

  אלטרנטיבית, ‎\(P(X \geq t \EX[X]) \leq \frac{1}{t}\) ‏\end{claim}
‏\begin{claim}[א"ש צ'בישב] אם ‎\(X\geq 0\) מ"מ עם תוחלת ושונות סופיים
  אז לכל ‎\(t>0\) מתקיים ‎\( P(\abs{X - \EX[X]} \geq t) \leq \frac{\Var(X)}{t^2} \)
  ‏\end{claim}
‏ \begin{claim}[א"ש צ'בישב חד צדדי]
‏
‏\end{claim}
‏\begin{theorem}[החוק החלש של המספרים הגדולים]
אם ‎\(X_1, \dots ,X_n\) מ"מ ב"ת שווי הסתברות עם שונות סופית, אז לכל ‎\(\epsilon >0\) מתקיים

‎\[ P\left(\abs{\frac{1}{n} \sum_{i=1}^n X_i - \EX[x_1]} > \epsilon\right) \overset{n\to \infty}{\longrightarrow} 0 \]
‏\end{theorem}


\ifxetex{\centering
  \begin{sidewaystable}
% \centering
  \begin{tabular}{ p{3cm} | p{4cm} | c | c | c | c | c}
    התפלגות  & הסבר
    & $\EX$ & $\EX^2$ & $\Var$ & $P(x=k) =$ & $\supp$ \\
    \hline
    אחידה \hfill‏ ‎\(\Uniform(n)\)
             & הסתברות שווה לכל מספר מ1 עד ‎\(n\)
    & ‎\(\frac{n+1}{2}\)
            & ‎\(\nicefrac{(n+1)(2n+1)}{6}\)
                      & ‎\(\frac{n^2 -1}{12}\)
                               & ‎\(\nicefrac{1}{n}\)
                                            & ‎\(\{1, \dots ,n\}\)\\

    ברנולי \hfill‏ ‎\(\Ber(p)\)
             & הצלחה בסיכוי ‎\(p\), כשלון בסיכוי ‎\(1-p\)
    & ‎\(p\)
            & ‎\(p\)
                      & ‎\(p(1-p)\)
                               & ‎\(P(X\!\!=\!\!0) = 1-p,\  P(X\!\!=\!\!1) = p\)
                                            & ‎\(\{0,1\}\) \\
    בינומית \hfill‏ ‎\(\Bin(n,p)\)
             & סכום של ‎\(n\) ניסויים ב"ת, כשכל אחד מהם מצליח בהסתברות ‎\(p\)
    & ‎\(n p\)
            & ‎\(n p (1-p) + n^2 p^2\)
                      & ‎\(n p (1-p)\)
                               & ‎\(\binom{n}{k}p^k (1-p)^{n-k}\)
                                            & ‎\(\{0, \dots ,n\}\) \\
    בינומית שלילית \hfill‏ ‎\(\NB(r,p)\)
             & כמות הטלות המטבע עם עץ בהס\-תברות ‎\(p\) עד שקיבלנו ‎\(r\) עצים (כולל)
    & ‎\(\frac{pr}{1-p}\)
            & ‎\(\frac{pr(1+pr)}{(1-p)^2}\)
                      & ‎\(\frac{pr}{(1-p)^2}\)
                               &  \(\binom{k|r-1}{k}(1-p)^r p^k\)
                                            & \(\NN_{0+}\) \\
    גאומטרית \hfill‏ ‎\(\Geom(p)\)
             & מטילים מטבע עם הסתברות ‎\(p\) לעץ וסופרים הטלות על עד הראשון (כולל)
    & ‎\(\nicefrac{1}{p}\)
            & ‎\(\nicefrac{1}{p^2}\)
                      & ‎\(\frac{1-p}{p^2}\)
                               & ‎\((1-p)^{k-1}p\)
                                            & ‎\(\NN_{1+}\) \\
    \parbox{3cm}{
    היפר גאומטרית\\
      \hspace*{\fill} \(\HG(N,D,n)\)}
             & דוגמים ‎\(n\) פריטים מאוכלוסיה בגודל ‎\(N\) שבה יש ‎\(D\) מיוחדים.
               \newline
               כמה מיוחדים מצאנו?
    & ‎\(\frac{Dn}{N}\)
            & ‎\(\frac{D(N-D)(N-n)n}{N^2(N-1)} + \frac{D^2n^2}{N^2}\)
                      & ‎\(\frac{D(N-D)(N-n)n}{N^2(N-1)}\)
                               & ‎\(\binom{D}{k}\binom{N-D}{n-k}\big{/}\binom{N}{n}\)
                                            & ‎\(\begin{array}{ll} \{\max(0,n+D),\\
                                                  \hfill\dots,\min(n,D)\}
                                                \end{array}\) \\

    פואסון \hfill‏ ‎\(\Poi(\lambda)\)
             & מספר הצלחות בתהליך בינומי כאשר ‎\(n\to \infty , p\to 0, n p\to \lambda \)
    & ‎\(\lambda \)
            & ‎\(\lambda + \lambda^2\)
                      & ‎\(\lambda\)
                               & ‎\(e^{-\lambda} \frac{\lambda^k}{k!}\)
                                            & ‎\(\NN_{0+}\)
  \end{tabular}
\end{sidewaystable}}\fi






\end{document}
